% !TEX root = ../lab5.tex
\section{Техническое задание}

\begin{itemize}
	\item Подобрать видео с движущимся объектом;
	\item Обозначить координаты объекта на первом кадре;
	\item Определить координаты объекта на последующих кадрах с использованием векторов смещения;
	\item Убедиться в корректности найденного объекта с помощью сравнения по гистограмме и отобразить кадры с обнаруженным (или потерянным) объектом.
\end{itemize}

\section{Ход работы}

Для выполнения работы были разработаны классы ObjectRectangle (обозначение координат объекта на первом кадре) и Tracker (определение координат объекта на последующих кадрах) с использованием библиотек OpenCV и Matplotlib (листинг \vref{lst:tracker}), а также классы графического интерфейса MainWindow (основное окно) и VideoWindow (простейший видеоплеер) с использованием библиотеки PyQt5 (листинг \cref{lst:ui}) на языке Python. Использованные зависимости для virtualenv приведены в листинге \cref{lst:requirements}.

\subsection{Описание алгоритма}

\begin{enumerate}
	\item Определяем координаты объекта (NxM);
	\item Для каждого последующего кадра видео производится обход некоторой окрестности объекта (2Nx2M) в предыдущем кадре в поиске максимального соответствия (с помощью метрик SAD, SSD);
	\item Таким образом, мы получаем смещения объекта, указывающие его "`движение"' между кадрами. С их помощью устанавливаем координаты текущего расположения объекта.
	\item Корректность найденного смещения проверяется с помощью сравнения гистограмм объекта предыдущего кадра и текущего (допускается совпадение не менее 50\% гистограммы).
\end{enumerate}

\begin{equation}
	SAD(d_1,d_2) = \sum_{i=0}^{n1}\sum_{j=0}^{n2}(d_1[i,j] - d_2[i,j])
\end{equation}

\begin{equation}
	SSD(d_1,d_2) = \sum_{i=0}^{n1}\sum_{j=0}^{n2}(d_1[i,j] - d_2[i,j])^2
\end{equation}

\begin{equation}
	HI = \sum_{i=0}^{255}\min(I_{t-1}^i, I_{t}^i)
\end{equation}

\subsection{Работа приложения}

Работа приложения демонстрируется на двух видеозаписях, начальные кадры которых приведены на \vrefrange{pic:1_1}{pic:2_1}, а объект выделен черной рамкой. Приложение отслеживало объект в течение 20-ти кадров (если объект не был потерян).

\begin{figure}[H]
	\centering
	\includegraphics[width=\textwidth]{1_1}
	\caption{Пример №1: первый кадр}
	\label{pic:1_1}
\end{figure}

\begin{figure}[H]
	\centering
	\includegraphics[width=\textwidth]{2_1}
	\caption{Пример №2: первый кадр}
	\label{pic:2_1}
\end{figure}

Приведем некоторые кадры для каждого из примеров (для примера №1 --- \vrefrange{pic:1_2}{pic:1_4}, для примера №2 --- \vrefrange{pic:2_2}{pic:2_3}).

\begin{figure}[H]
	\centering
	\includegraphics[width=\textwidth]{1_2}
	\caption{Пример №1: 12-ый кадр}
	\label{pic:1_2}
\end{figure}

\begin{figure}[H]
	\centering
	\includegraphics[width=\textwidth]{1_3}
	\caption{Пример №1: 16-ый кадр}
	\label{pic:1_3}
\end{figure}

\begin{figure}[H]
	\centering
	\includegraphics[width=\textwidth]{1_4}
	\caption{Пример №1: 20-ый кадр}
	\label{pic:1_4}
\end{figure}

\begin{figure}[H]
	\centering
	\includegraphics[width=\textwidth]{2_2}
	\caption{Пример №2: 5-ый кадр}
	\label{pic:2_2}
\end{figure}

\begin{figure}[H]
	\centering
	\includegraphics[width=\textwidth]{2_3}
	\caption{Пример №2: 6-ой кадр (объект потерян)}
	\label{pic:2_3}
\end{figure}

В случае кадра на \vref{pic:2_3} объект был потерян из-за его удаленности от области поиска (объект находится в ней лишь частично).

\section{Выводы}

В данной работе было проведено отслеживание движения объекта с использованием метрики максимального соответствия и проверкой корректности обнаруженного объекта с помощью сравнения гистограмм.

В результате было обнаружено, что полученная реализация весьма качественно находит двигающийся объект. Однако в случае его быстрого перемещения установленного окна поиска в 2Nx2M оказывалось недостаточно для обнаружения. Это можно исправить увеличением окна поиска, но вместе с этим возрастет время поиска объекта и шанс обнаружения некорректного объекта.

В случае обнаружения некорректных объектов, отсеять их помогает сравнение гистограмм, но данный метод плохо работает при сильно деформирующихся объектах. В такой ситуации необходимо понизить границу допустимого совпадения гистограмм, однако вместе с этим растет риск обнаружения ложных объектов.

Таким образом, качество работы отслеживания объекта зависит от окна поиска и границы допустимого совпадения гистограмм, которые невозможно настроить универсально. Поэтому их настройка должна проводиться в зависимости от условий, в которых проводится отслеживание.